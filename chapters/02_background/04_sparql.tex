\section{SPARQL}
\label{sec:sparql}

\ac{SPARQL} is a query language developed by \ac{W3C} retrieve and manipulate \ac{RDF} graph content on the Web or in a \ac{RDF} store. A \ac{SPARQL} query contains a set of triple patterns called \textit{basic graph pattern}. These patterns are like \ac{RDF} triples except that subjects, predicates and objects may be replaced by variables. The basic graph pattern matches a sub-graph of the \ac{RDF} data and returns a new \ac{RDF} graph in which the variables are replaced with the matched data. Queries are usually processed by an \ac{HTTP} service, called \textit{\ac{SPARQL} endpoint}. \cite{world2013sparql}. The example below shows a \ac{SPARQL} query on DBpedia \ac{SPARQL} endpoint,\footnote{\url{https://dbpedia.org/sparql}} while Table \ref{tab:sparql-example} shows its result.

\begin{verbatim}
PREFIX dbr: <http://dbpedia.org/page/>
PREFIX dbo: <http://dbpedia.org/ontology/>
PREFIX dbp: <http://dbpedia.org/property/name/>

SELECT ?relative ?name WHERE {
    dbr:Jotaro_Kujo dbo:relative ?relative .
    ?relative dbp:name ?name .
}
\end{verbatim}

\begin{table}[!ht]
    \centering
    \begin{tabular}{|l|l|}
        \hline
        \multicolumn{1}{|c|}{\textbf{relative}} & \multicolumn{1}{c|}{\textbf{name}} \\ \hline
        <http://dbpedia.org/resource/Dio\_Brando> & "Dio Brando"@en \\ \hline
        <http://dbpedia.org/resource/Joseph\_Joestar> & "Joseph Joestar"@en \\ \hline
        <http://dbpedia.org/resource/Jonathan\_Joestar> & "Jonathan Joestar"@en \\ \hline
    \end{tabular}
    \caption{A query result example from DBpedia.}
    \label{tab:sparql-example}
\end{table}

\ac{SPARQL} queries supports features like union of patterns, nesting queries, optional patterns or filtering values. Once the \ac{RDF} sub-graph is computed, it's also possible to modify it by ordering, limiting and grouping the values.

Another important feature of \ac{SPARQL} is the possibility to perform federated queries, which explicitly delegates certain sub-queries to different \ac{SPARQL} endpoints, allowing to navigate through the Web of Data.

Finally, to return a more machine-readable form, \ac{SPARQL} supports four common exchange formats, which are: \ac{XML}, \ac{JSON}, \ac{CSV}, and \ac{TSV} \cite{world2013sparql}.