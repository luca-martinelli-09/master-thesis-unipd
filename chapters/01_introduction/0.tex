\chapter{Introduction}
\label{chp:introduction}

\acl{LOD} will help transform eGovernment by enabling public administrations to define relationships between data of others. In addition, publishing \acl{LOD} creates new knowledge and encourages creativity and innovation. Indeed, governments can activate smarter and more efficient public services and applications, and organizations and citizens can develop new applications and tools in order to work with, analyze, and make sense of the data \cite{isa2013lod}.

In this context, the \ac{OntoIM} has the aim to extend the Italian OntoPiA ontology in order to describe the main semantic areas of a city, enabling it to publish the data as \acl{LOD}. Such areas are: Accommodation Facilities, Addresses and Civic Numbers, Cultural Heritage, Demographic Observations and Events, Facilities and Cadastral Data, Green Zones and Plants, Hospitals, IoT Events, Organizations and Associations, Point of Interests, Public Events, Public Services, Roads and Traffic, Schools, Transparency, Projects, and Public Contracts, Waste Production.

The \ac{OntoIM} ontology will also allow to link the data with other resources on the web, from national agencies, and from other cities.

In addition to the \ac{OntoIM} ontology, this thesis has the purpose to provide tools and web applications with the purpose of simplifying on the one hand the publication of Open Data and \acl{LOD} by public administrations and municipalities, and on the other hand making those data accessible through the use of data visualization tools.

The project would have benefits: (1) \textbf{for citizens}, who will be able to more easily access the data produced by the public administration, better understand their local area, and obtain information on events or accommodations; (2) \textbf{for businesses}, which will have data on the area that they can analyze, for example, to make targeted marketing plans; and (3) \textbf{for the public administration}, which would have data enriched with semantic meaning through the use of ontology, have the opportunity to be more transparent, could use the data to make and plan more targeted projects, and provide smarter and more efficient public services and applications.

Finally, a use case with data from the Comune di Sona\footnote{\url{https://comune.sona.vr.it}} will also be shown in this thesis. In fact, this project will be part of the \textit{Innovation Lab}\footnote{\url{https://innovationlab.regione.veneto.it/}} project, a project funded by the Regione Veneto that aims to spread digital and Open Data culture.

\subimport{}{01_scope}%