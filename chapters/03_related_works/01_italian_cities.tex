\section{Italian cities}
\label{sec:italian-cities}

For what concerns Italian cities, has been analyzed the most economically and culturally relevant cities in northern, central, and southern Italy: Bologna, Firenze, Genova, Milano, Napoli, Roma, Torino, and Venezia. The results, collected during April 2022, are shown in Table \ref{tab:italian-cities}. All the cities has obtained a score of three stars, since data are mostly published in non-proprietary format, in particular \ac{CSV}, \ac{JSON}, and Shapefile. Firenze and Bologna use \acs{API} that serves the resources in different formats. Firenze's data can be accessed in \acs{JSON} format or downloaded as a ZIP archive containing the \ac{CSV} file and a metadata file. Bologna, on the other hand, lets export resources in different formats, including \ac{RDF}/\ac{XML}, \ac{JSON}-LD, N-Triples, and Turtle. However, these resources are not accessible through \ac{SPARQL}, there are no semantic information, and they're not linked each other. For these reasons this catalog also obtained three stars.

\begin{table}[!ht]
  \onehalfspacing
  \centering
  \begin{tabular}{|l|l|l|l|l|}
    \hline
    \multicolumn{1}{|c|}{\textbf{City}} & \multicolumn{1}{c|}{\textbf{\# Datasets}} & \multicolumn{1}{c|}{\textbf{Most common file type}} & \multicolumn{1}{c|}{\textbf{Score}} & \multicolumn{1}{c|}{\textbf{Software}} \\ \hline
    Firenze                   & 1902                                             & Uses \acs{API}                                       & 3                                   & Drupal + CKAN \\ \hline
    Bologna                   & 425                                              & Uses \acs{API}                                       & 3                                   & OpenDataSoft  \\ \hline
    Milano                    & 1618                                             & CSV (1540)                                           & 3                                   & CKAN  \\ \hline
    Torino                    & 1954                                             & CSV (1460)                                           & 3                                   & CKAN  \\ \hline
    Roma                      & 319                                              & CSV (230)                                            & 3                                   & CKAN  \\ \hline
    Venezia                   & 248                                              & CSV (179)                                            & 3                                   & Drupal  \\ \hline
    Genova                    & 138                                              & CSV (111)                                            & 3                                   & DKAN  \\ \hline
    Napoli                    & 62                                               & CSV (35)                                             & 3                                   & Custom  \\ \hline
  \end{tabular}
  \caption{Analysis of Italian cities' Open Data Portals. The data reported in this table was collected during April 2022.}
  \label{tab:italian-cities}
\end{table}

All the Italian cities analyzed, except Napoli, follows the \textit{Linee guida nazionali per la valorizzazione del patrimonio informativo pubblico}.\footnote{\url{https://docs.italia.it/italia/daf/lg-patrimonio-pubblico/}} Indeed, they provide their entire catalog as \acl{LOD} using the \verb#DCAT_AP-IT# ontology for resource metadata, like the access and download \acs{URL}, the name and the file type of the resource, the owner of the dataset, the frequency of updating the data, the theme, and more. This approach aims to maintain the ease of publishing data (e.g. using the CKAN portal), but at the same time allows resources to be more accessible, provide additional information about the nature of the data, and enables the ability to access resources from regional, national,\footnote{\url{https://dati.gov.it}} and European\footnote{\url{https://data.europa.eu}} portals.

\paragraph*{}
Finally, some notable attempts to publish data in \acl{LOD} format come from Milan and Bologna, which have respectively two portals (\url{https://dati.comune.milano.it/sparql/home.html}, and \url{http://linkeddata.comune.bologna.it}) dedicated to \acl{LOD}. Milano developed a custom ontology called \textit{OntoMI}\footnote{\url{https://dati.comune.milano.it/sparql/onthdoc.html}} that partially extends OntoPiA, which is described in Section \ref{sec:ontopia}. In particular, the ontology describes six subject areas that represent a part of the services offered by the City of Milano: libraries, administrative acts, kindergartens, consumer price detection, sports facilities, and Area C entry detection. However, the \ac{SPARQL} endpoint is no longer available, and the data can no longer be accessed. For what concerns Bologna, it also developed a custom ontology, called \textit{Onto Municipality}\footnote{\url{http://linkeddata.comune.bologna.it/ontologies/2014/04/onto-municipality/}}, that describes districts, areas, streets, squares and other circulation areas, civic numbering, places and people of interest, schools, and demographic statistics. For the latter, Bologna uses an ontology developed by ISTAT as part of the 2011 census\footnote{\url{https://www.istat.it/it/archivio/160039}} that is no longer maintained and accessible. Despite the \ac{SPARQL} endpoint, and the data are still accessible, the project has not been maintained since 2016, making it currently useless.