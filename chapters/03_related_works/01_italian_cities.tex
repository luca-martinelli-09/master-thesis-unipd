\section{Italian cities}
\label{sec:italian-cities}

For what concerns Italian cities, has been analyzed the most economically and culturally relevant cities in northern, central, and southern Italy: Bologna, Firenze, Genova, Milano, Napoli, Roma, Torino, and Venezia. The results, collected during April 2022, are shown in Table \ref{tab:italian-cities}. All the cities has obtained a score of three stars, since data are mostly published in non-proprietary format, in particular \ac{CSV}, \ac{JSON}, and Shapefile. Firenze and Bologna use \acs{API} that serves the resources in different formats. Firenze's data can be accessed in \acs{JSON} format or downloaded as a ZIP archive containing the \ac{CSV} file and a metadata file. Bologna, on the other hand, lets export resources in different formats, including \ac{RDF}/\ac{XML}, \ac{JSON}-LD, N-Triples, and Turtle. However, these resources are not accessible through \ac{SPARQL}, there are no semantic information, and they're not linked each other. For these reasons this catalog also obtained three stars.

\begin{table}[!ht]
  \onehalfspacing
  \centering
  \begin{tabular}{|l|l|l|l|}
    \hline
    \multicolumn{1}{|c|}{\textbf{City}} & \multicolumn{1}{c|}{\textbf{\# Datasets}} & \multicolumn{1}{c|}{\textbf{Most common file type}} & \multicolumn{1}{c|}{\textbf{Score}} \\ \hline
    Firenze                   & 1902                                             & Uses \acs{API}                                       & 3                                   \\ \hline
    Bologna                   & 425                                              & Uses \acs{API}                                       & 3                                   \\ \hline
    Milano                    & 1618                                             & CSV (1540)                                           & 3                                   \\ \hline
    Torino                    & 1954                                             & CSV (1460)                                           & 3                                   \\ \hline
    Roma                      & 319                                              & CSV (230)                                            & 3                                   \\ \hline
    Venezia                   & 248                                              & CSV (179)                                            & 3                                   \\ \hline
    Genova                    & 138                                              & CSV (111)                                            & 3                                   \\ \hline
    Napoli                    & 62                                               & CSV (35)                                             & 3                                   \\ \hline
  \end{tabular}
  \caption{Analysis of Italian cities' Open Data Portals. The data reported in this table was collected during April 2022.}
  \label{tab:italian-cities}
\end{table}

Finally, all the Italian cities analyzed, except Napoli, provide their entire catalog as \acl{LOD} using the \verb#DCAT_AP-IT# ontology for resource metadata, like the access and download \acs{URL}, the name and the file type of the resource, the owner of the dataset, the frequency of updating the data, the theme, and more. This approach aims to maintain the ease of publishing data (e.g. using the CKAN portal), but at the same time allows resources to be more accessible, provide additional information about the nature of the data, and enables the ability to access resources from regional, national,\footnote{\url{https://dati.gov.it}} and European\footnote{\url{https://data.europa.eu}} portals.