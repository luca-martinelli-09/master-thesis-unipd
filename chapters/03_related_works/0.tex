\chapter{Related works}
\label{chp:related}

As said in Chapter \ref{chp:introduction}, the purpose of this thesis is to design an ontology for Italian municipalities, facilitate the publication on the Web of \acl{LOD}, and develop a web application that makes this data easier for people to comprehend and visualize. It is therefore interesting to understand how major Italian, European, and global cities publish their data on the Web, and what data they publish. 

The next sections will show the information collected by Italian, European and global cities about their Open Data portal, and in particular: (1) the number of available datasets; (2) the most common data file types; (3) a score from 1 to 5 based on the five stars classification presented in Section \ref{sec:lod-stars}. Since the score is assigned to the entire data catalog and not to a single resource, only the types of files most present in the portal were considered.

\subimport{}{01_italian_cities}%
\subimport{}{02_other_cities}%

\paragraph*{}
To conclude the analysis on Italian, European and global cities' approach to Open Data, we can definitely see that no cities publish \acl{LOD}, but they prefer publish resources in using non-proprietary (and in some cases proprietary) format, reaching a score of three or fewer stars. This is probably due to the fact that convert and publishing data as \acl{LOD} has a greater cost in terms of time and economic resources \cite{bauer2011linked}. The examples of Milano and Bologna, which have stopped investing in \acl{LOD}, are proof of this. However, these costs can be covered by states, ministries, government institutions, or regions, which instead publish a portion of their data as \acl{LOD}. Some examples are the Europeana project,\footnote{\url{https://www.europeana.eu}}, Ministero della Cultura,\footnote{\url{https://dati.cultura.gov.it/}} ISPRA,\footnote{\url{http://dati.isprambiente.it/}} Regione Veneto,\footnote{\url{https://www.culturaveneto.it/it/}} or Regione Sicilia.\footnote{\url{https://dati.regione.sicilia.it/i-linked-open-data-nel-catalogo-regionale/}}

Finally, of interest is the approach of Italian and EU cities in publishing the catalog in \ac{RDF} format using the DCAT metadata profile, which allows them to provide some semantic information to the datasets, such as the owner of the data, the frequency of update, or the topic to which the data refer.