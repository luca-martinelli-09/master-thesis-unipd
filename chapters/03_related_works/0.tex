\chapter{Related works}
\label{chp:related}

As said in Chapter \ref{chp:introduction}, the purpose of this thesis is to design an ontology for Italian municipalities, facilitate the publication on the Web of \acl{LOD}, and develop a web application that makes this data easier for people to comprehend and visualize. It is therefore interesting to understand how major Italian, European, and global cities publish their data on the Web, and what data they publish. 

The next sections will show the information collected by Italian, European and global cities about their Open Data portal, and in particular: (1) the number of available datasets; (2) the most common data file types; (3) a score from 1 to 5 based on the five stars classification presented in Section \ref{sec:lod-stars}. Since the score is assigned to the entire data catalog and not to a single resource, only the types of files most present in the portal were considered.

\subimport{}{01_italian_cities}%
\subimport{}{02_other_cities}%

%% recap lod

%% città italiane

%% città europee ed extraeuropee

%% considerazioni