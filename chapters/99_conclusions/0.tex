\chapter{Conclusions and Future Works}
\label{chp:conclusions}

The goal of this thesis is to document the design and the development of the first version of the \ac{OntoIM} ontology, the \ac{RDF} Graph Builder, and the \textit{Data Reports} and CKAN web applications. The goal is to simplify and encourage the adoption by Italian municipalities of technologies to publish data produced by the public administration in \acl{LOD} format, and make it more understandable for citizens, businesses, and the administration itself to consult this data.

The \ac{OntoIM} ontology was designed as an extension of OntoPiA, the Italian ontology developed by \ac{AgID}, Italian Digital Transformation Team, in cooperation with other national agencies. In particular, the \ac{OntoIM} was designed in order to describe the data collected from the Comune di Sona, Agenzia delle Entrate, Camera di Commercio, ISPRA, ISTAT, Ministero dell'Istruzione, and after an analysis on the data published by Italian, and global cities. The result is an ontology that describes different domains of a city, such as demographic observations, schools, hospitals, waste production, private organizations, associations, \etc., so that it is possible to describe as many aspects of a city as possible, and to be able to publish that data in \acl{LOD} format. The first version of the \ac{OntoIM} ontology, and its documentation, can be found at \url{https://w3id.org/ontoim}.

The second part of the project was the development of two Python libraries, \verb#ontopia-py#,\footnote{\url{https://github.com/luca-martinelli-09/ontopia-py}} and \verb#ontoim-py#.\footnote{\url{https://github.com/luca-martinelli-09/ontoim-py}} These libraries allow an \ac{RDF} Graph to be created using the object-oriented programming paradigm, converting ontology classes to Python classes, and properties to attributes. This makes it possible to insert resources more easily, resulting in more readable code, and following ontology rules. These libraries were used to build the \ac{RDF} Graph of the Comune di Sona by importing data from three different sources: the CKAN Open Data portal, from offline files, and government \acsp{API}. The \ac{RDF} Builder for the Comune di Sona is available on GitHub at \url{https://github.com/luca-martinelli-09/sona-lod}.

The third and final part of the project was the development of two web applications. The first one is the configuration of a CKAN Open Data portal for the Italian municipalities that follows the \textit{Linee guida nazionali per la valorizzazione del patrimonio informativo pubblico}.\footnote{\url{https://docs.italia.it/italia/daf/lg-patrimonio-pubblico/}} The second one is \textit{Data Reports} a web application built from scratch, using the Svelte JavaScript framework. \textit{Data Reports} has been designed as an online journal where each article can contain text, and charts, maps, and tables built using the data retrieved by \ac{SPARQL} queries. The goal is to have a portal where citizens, businesses and the administration can consult data in an easy and understandable way, with comments on the data and results. The two projects are available respectively at \url{https://github.com/luca-martinelli-09/ckan}, and \url{https://github.com/luca-martinelli-09/ontoim-webapps}.

\paragraph*{}

As future works, the \ac{OntoIM} ontology will be subdivided into different ontologies, one for each semantic area, to follow the principle of modularity of the Italian OntoPiA ontology, and with the goal of becoming part of it.

For what concerns the \ac{RDF} Builder, the future project is to develop a universal and easy-to-setup mapper to convert data from different sources into \acl{LOD}, according to the \ac{OntoIM} and OntoPiA ontologies.

Finally, the \textit{Data Reports} web application will be improved so that it can be maintained through a private area that allows adding and creating charts directly from the web interface.